% The main LaTeX file for the NexusWide StyleGuide git_project_management PDF documentations.


%
\documentclass[13pt]{scrarticle}
%


% text_formatting_oriented_packages packages.
\usepackage[utf8]{inputenc}
\usepackage[english]{babel}
\usepackage{microtype}
\usepackage[none]{hyphenat}
\usepackage{parskip}
\usepackage[onehalfspacing]{setspace}
%


% graphical_formatting_oriented packages.
\usepackage{graphicx}
\usepackage{fancyhdr}
\usepackage{xcolor}
\usepackage{tikzsymbols}
\usepackage{hyperref}
%


%
\newcommand{\header}[1]{ \textsf{#1} \relax{}}
\newcommand{\important}[1]{\textit{#1}}
\newcommand{\name}[1]{{\textsc{#1}}}
\newcommand{\nexusrule}[1]{\Tribar[2][white][yellow][brown]{\color{brown}\hspace{0.5cm}#1}}
%


%
\setlength{\parindent}{0cm}
\widowpenalties 1 1000
\clubpenalties 1 1000
\interlinepenalty 1000


\renewcommand{\familydefault}{\sfdefault}
\renewcommand{\headrulewidth}{0.1cm}
%


%
\title{NexusWide Project Managemet Guidelines}
\author{Nima Bavar}
\date{\today}
%


\begin{document}
    \raggedright
    \pagestyle{fancy}


    \fancyhf{}
    \pagestyle{fancy}
    \lhead{\leftmark}
    \rhead{\thepage}

    \pagenumbering{gobble}


    \thispagestyle{empty}
    \pagecolor{white}


    \begin{figure}[t]
        \centering \includegraphics[width=14cm]{/home/nimabavar/Desktop/work/diary/projects/nexus_wide/repositories/style_guide/attachments/nexus_wide_logo.png}
    \end{figure}

    \centering{\name{\hspace{2cm} Research Institute Presents.}} \newline


    \thispagestyle{empty}
    \maketitle{}


    \newpage
    \thispagestyle{fancy}
    \pagenumbering{gobble}
    \setcounter{page}{2}


    \section*{\header{Copyright}}
    \thispagestyle{empty}

    \raggedright
    The corresponding documentation literature and all the containment's, are certified under the \name{Creative Commons Zero v1.0 Universal License} ,
    with the elucidation of the actuality that all branches of \name{modification}, \name{distribution}, \name{patent}, \name{commercial } and \name{private usage } are entirely authorized.
    \newline

    This fragment of dossier is submitted \important{AS IS},
    you are liberate to manufacture any category of modifications you find essential for your individuality or organization.
    \newline

    However, this aforementioned fact loose the \name{NexusWide } team from any fork of provisioned liability or warranty regarding the usage scenario of such composition,
    denoting that we are not held accountable for whichever harm caused by the employment/misemployment of this attestation in any form,
    therefore we demand you to cautiously study the above-mentioned license sanctions if the decision for calibrating this file in a
    personal or officially concealed environment was made. \newline

    The \name{NexusWide } research team welcomes all enhancement ideations announced or pushed to the official repository of the imminent writing.
    \newline


    \newpage
    \thispagestyle{empty}
    \pagenumbering{Roman}


    \begin{centering}
        \vspace*{7cm}
      
        \hspace{1.0cm}
        Dedicated to all the computation enthusiastics \newline
        \hspace*{2.5cm}
        and \newline
        \hspace{0.3cm}
        \name{Free and Open Source Software } producers around the globe.

    \end{centering}


    \newpage
    \pagenumbering{roman}
    \thispagestyle{fancy}
    \setcounter{page}{4}


    \tableofcontents


    \newpage
    \thispagestyle{fancy}
    \pagenumbering{arabic}
    \setcounter{page}{1}

    \section{\header{Conventions}}

    Numerous conventions are globally used throughout this document
    in order to draw attention or to cite particular meanings,
    which are stated in the posterior table:


    \vspace*{2cm}
    \begin{tabular}{r l}

        \hspace{2.8cm}
        \raggedright \textbf{\Large Emphasization} & \textbf{\Large definition} \tabularnewline

        \nexusrule{Brown text with Tribar} & Nexusrule \tabularnewline
        \name{small caps text} & Name \tabularnewline
        \important{Italic text} & Important Signification

    \end{tabular}
    \vspace*{2cm}

    The acknowledgment of the aforementioned elements shall be enough
    for you to be prepared to start reading this documentation. \newline

    Do not hesitate to reach out to this table anytime you have forgave the reason behind a specific emphasizing regularity. \newline


    \newpage
    \section{\header{Preface}}


    \name{Durability } and \name{consistency } are the two most prominent chunk of a bested project management course,
    and no project is capable of achieving satisfactory outcomes without proper administration. \newline

    One branch of behavior that can further advance durability by obligating consistency in the best feasible manner,
    is including consequential regulations and laws,
    which is preserved by affixing sharp style guides to the association. \newline

    Thereby to achieve pleasing result in order to avail the esteem and sagacious community of \name{FOSS}\footnote{Free and Open and Source Software.} with our works of knowledge and researches,
    and most essentially, provide value to our respected readers and users,
    we shall possess our own particular style guides for the contributors to adhere. \newline

    The foundation of the following pertinent standards and rules, was gathered by the acknowledgment of the fore-mentioned
    principles, with notable absorption to the domain of project management. \newline

    We \important{oblige} all of our contributors to adhere to them,
    in order to acquire accommodate and flaw-free products. \newline

    Please reach out to the succeeding pages,
    and interpret them precisely.


    \newpage
    \thispagestyle{fancy}

    \section{\header{Rules}}

      \nexusrule{All \name{NexusWide } project repositories MUST contain a 'docs' directory.} \newline\nopagebreak

      \nexusrule{All \name{NexusWide } project repositories MUST have a 'README.md' file located in their 'docs' directory which follows the Markdown formatting and style guidelines.} \newline\nopagebreak

      \nexusrule{All \name{NexusWide } project repositories MUST contain a 'Security Policy' file.} \newline\nopagebreak

      \nexusrule{All of the following segments: 'Security Policy', 'Code of Conduct', 'Contributing Guidelines', and 'Terms of Service' MUST be located within the 'docs' directory.} \newline\nopagebreak

      \nexusrule{Only the 'License' file is permitted to be located in the main directory of the repository.} \newline\nopagebreak

      \nexusrule{All \name{NexusWide } owned repositories MUST be licensed under \linebreak ( Creative Commons Zero v1.0 Universal ). } \newline\nopagebreak

      \nexusrule{Force commits are not allowed in \name{NexusWide } repositories.} \newline\nopagebreak

      \nexusrule{Fast-Forward merges are not allowed in \name{NexusWide } repositories.} \newline\nopagebreak

      \nexusrule{All versions of the project repository MUST be controlled under semantic versioning ( 'https://semver.org/' ).} \newline\nopagebreak

      \nexusrule{The version numbering of the project MUST be split into two phases: \newline( pre release phase, release phase ).} \newline\nopagebreak

      \nexusrule{ During the pre release phase all changes in the development process are a part of the version numbering.} \newline\nopagebreak

      \nexusrule{ The release version numbering MUST govern only the user side features.} \newline\nopagebreak

      \nexusrule{ When a project reaches the release phase, a ( changelog.md ) file MUST be implemented in the repository ( docs ) directory.} \newline\nopagebreak

      \nexusrule{ The release phase version numbering is only mentioned in the \linebreak ( changelog.md ) file.} \newline\nopagebreak

      \nexusrule{ Release versioning and development versioning are seperate concepts, do not attempt to increment the release version number while proposing updates to the back end side of the projects.}
      \newline

      \nexusrule{ Local branches MUST have their name set as the part of the API they are going to affect, followed by \newline the branch number:\newline ( The branch number must be used in order to avoid conflicts with the previous remote branches of the same name ) \newline ( e.x: documentations1 ).} \newline\nopagebreak

      \newpage
      \nexusrule{ Contributors MUST always create a pull request and branch before \nopagebreak attempting to make their local changes and submit an update.}

      \nexusrule{ Remote branches MUST be deleted after they have been merged into the master ( Main.Project ) branch in any form.} \newline\nopagebreak

      \nexusrule{ Pull requests MUST NOT contain a to do list, as that would make the existence of issues purposeless.} \newline\nopagebreak

      \nexusrule{ Pull request titles MUST refer to the fix, feat or MAJOR change commit message which is expected to be the outcome after all the issue to todo tasks are finished.} \newline\nopagebreak

      \nexusrule{ the convention of naming pull requests is the same as commit messages.} \newline\nopagebreak
      \nexusrule{ The convention of naming pull requests is the same as commit messages.} \newline\nopagebreak

      \nexusrule{ All pull requests MUST have at least one comment before being merged.} \newline\nopagebreak

      \nexusrule{ Contributors MUST NOT directly list any of the ( fix ) or ( feat ) changes in the description comment of a pull request ( using bulleted or numbered lists, ETC ) They MUST be self explanatory in the commit messages.} \newline\nopagebreak

      \nexusrule{ All pull reqeusts MUST contain a well formatted description comment.} \newline\nopagebreak

      \newpage
      \nexusrule{ The pull request description comment MUST always be the first comment.} \newline\nopagebreak

      \nexusrule{ The pull request description comment MUST start with a header containing the update name tag \newline ( this is not the same as the pull request title, it is any name tag that you would want to assign ) followed by the version number. \newline ( e.x: documentation changes description header would be: ( UDock $\vert$ 1.0.0 ) ) .} \newline\nopagebreak

      \nexusrule{ The pull request description comment MUST contain four sections: What has changed ( API changes ). What were the reasons behind the changes. What results are expected. Which issue notes are affected by this update} \newline\nopagebreak

      \nexusrule{ The 'signed off by' section MUST be the last section of the description comment and refer to the contributors who parted in the update.} \newline\nopagebreak

      \nexusrule{ Issue references in pull requests MUST follow the following format: \newline ( Issue Note $\vert$ { Issue Number } ).} \newline\nopagebreak

      \nexusrule{ If a update has been rejected, The branch related to the update MUST be deleted, the pull request MUST be closed and labled as 'rejected' \newline ( The description comment MUST stay as is ).} \newline\nopagebreak

      \nexusrule{ All code MUST be reformatted and pass all the tests given by the GitHub Actions before being merged.} \newline\nopagebreak

      \newpage
      \nexusrule{ Issue names MUST follow the corresponding regular expression: \newline ( { Issue Title } $\vert$ { Issue numeral ccount } ), \newline ( e.x: problem(docs): remove harsh rules $\vert$ 3 ) .} \newline\nopagebreak

      \nexusrule{ All issues MUST contain at least 1 comment.} \newline\nopagebreak

      \nexusrule{ The first comment of an issue MUST be the  details  comment.} \newline\nopagebreak

      \nexusrule{ The issue  details  comment MUST be formatted by the following topics: \newline What is the assignment? \newline What are the steps to accomplish? \newline Why is this fixture necessary?  \newline  What is the expected outcome?} \newline\nopagebreak

      \nexusrule{ The issue  details  comment MUST contain a contributor name at its last section, in order to assign the contributor to the cited task.} \newline\nopagebreak

      \nexusrule{ The Git commit messages are formatted by the https://www.conventionalcommits.org/en/v1.0.37. conventions.} \newline\nopagebreak

      \nexusrule{ Git commits MUST be as small and independent as possible, do not concatenate two possible commits together \newline ( e.x: ( fix(docs): refactor grammar and add header ruleset ) \newline is a bad commit message .} \newline\nopagebreak

      \newpage
      \nexusrule{ Push changes as often as possible, all the changes to the project \linebreak ( including the ones that are rejected or/and aborted ) MUST be recorded.} \newline\nopagebreak

      \nexusrule{ All suspended projects ( those that don't receive any new update for a designated period of time ) MUST be archived and followed by a note text at the tail of their README file,        elucidating the fact that the project have been archived.} \newline\nopagebreak

      \nopagebreak \nexusrule{ All Git commit messages MUST have their branch name and the commit number \nolinebreak[4] ( the number of times that the contributor have committed to a branch ) as their footer \newline ( e.x: ( style guide: 3 ) ).} \newline\nopagebreak

      \nexusrule{ In the context of combining branches, use merge instead of rebase.} \newline\nopagebreak

      \nexusrule{ While performing any type of merging operation, adhere to the same guidelines as commit messages and provide the reason behind the merge \newline ( e.x: fetch(merge): fix conflict with origin main.project ) .} \newline\nopagebreak

      \nexusrule{ All remote repository merge messages MUST be the same as the title of their pull request.} \newline\nopagebreak

      \nexusrule{ All remote repository merge messages MUST have the URL of their pull request at their description section.} \newline\nopagebreak

      \newpage
      \nexusrule{ All merges messages ( Including the remote ones ) MUST have the 'merge' pseudo-environment count as their footer: ( The number of times that a merge operation was conducted in the repository ) \newline ( e.x: ( merge: 4 ) ).} \newline\nopagebreak

      \nexusrule{ All \name{NexusWide } repositories MUST have a Kanban board project derived from the \name{NexusWide } main Kanban board template.} \newline\nopagebreak

      \nexusrule{ The \name{NexusWide } Kanban board ( in\textunderscore progress and in\textunderscore review ) columns MUST be limited to one card only. \newline ( Only one update topic is allowed to be worked on at a time. )} \newline\nopagebreak[1]

      \nexusrule{ The \name{NexusWide } Kanban board ( wont\textunderscore fix, ready,  done ) columns MUST NOT introduce any card amount limitations.} \newline\nopagebreak

      \nexusrule{ All \name{NexusWide } card items MUST be converted to issues when they reach the ( ready ) state.} \newline\nopagebreak

      \nexusrule{ All \name{NexusWide } Kanban card names MUST adhere to the \name{NexusWide } Git commit message guidelines.} \newline\nopagebreak

      \nexusrule{ All \name{NexusWide } Kanban card names MUST be the name of the issue which is going to be produced after the item have reached \newline the ( in\textunderscore progress ) state, omitting the issue number.} \newline\nopagebreak[4]

      \newpage
      \nexusrule{ The issues generated from a Kanban card MUST have the same name as the card.} \newline\nopagebreak

      \nexusrule{ All \name{NexusWide } Kanban cards MUST have a description section aligned.} \newline\nopagebreak

      \nexusrule{ All \name{NexusWide } Kanban card descriptions MUST adhere to the issue details comment formatting guidelines.} \newline\nopagebreak

      \nexusrule{ All \name{NexusWide } Kanban card descriptions MUST be the description of the issue which is going to be produced after the item have reached the \newline( in\textunderscore progress ) state.} \newline\nopagebreak

      \nexusrule{ The issues generated from a Kanban card MUST have the same details comment content as the card description.} \newline\nopagebreak

      \nexusrule{ All \name{NexusWide } card items MUST be moved to the ( done ) column after the corresponding pull request have been merged.} \newline\nopagebreak

      \nexusrule{ All \name{NexusWide } card items MUST be moved to the ( wont\textunderscore fix ) column if the update have been rejected.} \newline\nopagebreak

      \nexusrule{ All \name{NexusWide } Kanban card items after/on the ( ready ) column MUST be assigned a start date.} \newline\nopagebreak

      \nexusrule{ All \name{NexusWide } Kanban card items after/on the ( ready ) column MUST be assigned a due date.} \newline\nopagebreak

      \newpage
      \nexusrule{ All \name{NexusWide } Kanban card items after/on the ( ready ) column MUST be assigned to a contributor ( assignee ).} \newline\nopagebreak

      \nexusrule{ All \name{NexusWide } Kanban card items after/on the ( ready ) column MUST be prioritized and labeled based on the tags offered by the \name{NexusWide } main Kanban template.} \newline\nopagebreak

      \nexusrule{ All NexusWide Kanban card items after/on the ( Ready ) column MUST be measured in size and labeled based on the tags offered by the NexusWide main Kanban template. }
      \newline\nopagebreak


\end{document}
